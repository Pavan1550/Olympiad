\documentclass{article}
\usepackage{gvv}
\begin{document}
\begin{enumerate}
		\subsection*{number theory}
\item problem 1 Prove that for any pair of positive integers $k$ and $n$, there exist $k$ positive integers $m_1,m_2,m_3,\ldots$
 (not necessarily different) such that
\begin{align}
	1+\frac{2^{k}-1}{n}=\brak{1+\frac{1}{m_1}}\brak{1+\frac{1}{m_2}}\ldots\brak{1+\frac{1}{m_k}}
\end{align}
		\subsection*{geometry}
\item Problem $2$. A configuration of $4027$ points in the plane is called Colombian if it consists of $2013$ red points and $2014$ blue points, and no three of the points of the configuration are collinear. By drawing some lines, the plane is divided into several regions. An arrangement of lines is good for a Colombian configuration if the following two conditions are satisfied:\\ 
 * no line passes through any point of the configuration;\\
* no region contains points of both colours.\\

Find the least value of $k$ such that for any Colombian configuration of $4027$ points, there is a good arrangement of $k$ lines
		\subsection*{geometry}

\item Problem 3. Let the excircle of triangle $ABC$ opposite the vertex $A$ be tangent to the side $BC$ at the point $A_1$. Define the points $B_1$, on $CA$ and $C_1$, on $AB$ analogously, using the excircles opposite $B$ and $C$. respectively. Suppose that the circumcentre of triangle $A_1B_1C_1$, lies on the circumcircle of triangle $ABC$. Prove that triangle $ABC$ is right-angled.

The excircle of triangle $ABC$ opposite the vertex $A$ is the circle that is tangent to the line segment $BC$, to the ray $AB$ beyond $B$, and to the ray $AC$ beyond $C$. The excircles opposite $B$ and $C$ are similarly defined.
		\subsection*{geometry}
\item problem4. Let $ABC$ be an acute-angled triangle with orthocentre$ H$, and let $W$ be a point on the side $BC$, lying strictly between $B$ and $C$. The points $M$ and $N$ are the fect of the altitudes from $B$ and $C$, respectively. Denote by $w_1$ the circumcircle of $BWN$, and let $X$ be the point on wy such that $WX$ is a diameter of $w_1$ Analogously, denote by $w_2$ the circumcircle of $CWM$. and let $Y$ be the point on such that $WY$ is a diameter of Prove that $X$, $Y$ and Hare collinear.
	\subsection*{number theory}
\item Problem 5. Let $ Q_{\textgreater 0}$ be the set of positive rational mumbers. Let $f: Q_{\textgreater 0} \rightarrow R$ be a function satisfying the following three conditions:
	\begin{enumerate}
		\item for all $x,y\epsilon  Q\textgreater0$, we have $f\brak{x} f\brak{y} \geq  f\brak{xy}$
		\item for all $x,y\epsilon Q\textgreater0$,we have$f\brak{x+y} \geq f\brak{x}+f\brak{y}$
		\item there exists a rational number $a\textgreater 1$ such that $f\brak{a}=a$.
			
		prove that $F\brak{x}=x$ for all $x \epsilon Q\textgreater0$.
       \end{enumerate}		
		\subsection*{combinatorics}
\item Problem 6. Let $n \geq  3$ be an integer, and consider a circle with $n+1$ equally spaced points marked on it. Consider all labellings of these points with the numbers $0,1,\ldots n$ such that each label is used exactly once, two such labellings are considered to be the same if one can be obtained from the other by a rotation of the circle. A labelling is called beautiful if, for any four labels $a\textless b\textless c\textless d$ with $a+d=b+c,$ the chord joining the points labelled $a$ and $d$ does not intersect the chord joining the points labelled $b$ and $c$ \\

Let $M$ be the number of beautiful labellings, and let $N$ be the number of ordered pairs $\brak{x, y}$ of
		positive integers such that $x+y\leq n and gcd \brak{x,y} = 1$. Prove that
                                                                
								$m=n+1$
\newpage 
		\subsection*{number theory}
\item problem1 let $a_{0\textless} a_{1\textless} a_{2 \textless} \ldots$ be an infinite sequence of positive integers.prove that there exists a unique integer $n\geq 1$such that\\
	\begin{align}
	a_{n\textless }\frac{a_0+a1+\ldots+a_n}{n} \textless a_{n+1}.
	\end{align}
		\subsection*{geometry}	
\item Problem 2. let $n\geq 2$ be an integer. Consider an $n\times n$ chessboard consisting of $n^2$ unit squares. A configuration of $n$ rooks on this board is peaceful if every row and every column contains exactly one rook. Find the greatest positive integer $k$ such that, for each peaceful configuration of $n$ rooks, there is a $k\times k$ square which does not contain a rook on any of its $k^2$ unit squares.
	\subsection*{geometry}
\item Problem 3. Convex quadrilateral $ABCD$ has $\angle ABC= \angle CDA = 90 \degree$ Point His the foot of the perpendicular from A to BD. Points S and T lie on sides $AB and AD$, respectively, such that $H$ lies inside triangle $SCT$ and $\angle CHS- \angle CSB = 90 \degree , \angle THC- \angle DTC = 90\degree$ .\\
	Prove that line $BD$ is tangent to the circumcircle of triangle $TSH$.
		\subsection*{geometry}
\item Problem 4. Points $P and Q$lie on side $BC$ of acute-angled triangle $ABC$ so that $\angle PAB= \angle BCA$ and $\angle CAQ=\angle ABC.$ Points $M$ and $N$ lie on lines $AP$ and $AQ,$ respectively, such that $P$ is the midpoint of $AM,$ and $Q$ is the midpoint of $AN.$ Prove that lines $BM and CN$ intersect on circumcircle of triangle $ABC$
	\subsection*{number theory}

\item Problem 5. For each positive integer $n,$ the Bank of Cape Town ienes coins of denomination$\frac{1}{n}$ Given a finite collection of such coins (of  not  necessarily  different  denominations) with total value at most $99 +\frac{1}{2}$ prove that it is possible to split this collection into $100$ or fewer groups, such that each group has total value at most $1$.
	\subsection*{geometry}
\item Problem 6. A set of lines in the plane is in general position if no two are parallel and no three pass through the same point. A set of lines in general position cats the plane into regions, some of which have finite area; we call these its finite regions. Prove that for all sufficiently large $n$. in any set of a lines in general position it is possible to colour at least $\sqrt n$ of the lines blue in such a way that none of its finite regions has a completely blue boundary.\\

Note: Results with $\sqrt n$ replaced by $c \sqrt n$  will be awarded points depending on the value of the constant $c$.

\newpage
		\subsection*{geometry}
\item Problem 1. We say that a finite set $S$ of points in the plane is balanced if, for any two different points $A and B$ in $S$, there is a point Cin Ssuch that $AC=BC$. We say that $S$ is centre-free if for any three different points $A, B$ and $C$ in $S$, there is no point $P$ in $S$ such that $PA=PB=PC$
	\begin{enumerate}

		\item  Show that for all integers $n\geq3$, there exists a balanced set consisting of $n$ points.

		\item  Determine all integers $n\geq3$ for which there exists a balanced centre-free set consisting of $n$ points.\\
	\end{enumerate}		
	\subsection*{geometry}
\item Problem 2. Determine all triples $\brak{a, b, c}$ of positive integers such that each of the numbers\\
			 $ ab-c, bc-a,ca-b$\\
		is a  power of $2$\\(A power of 2 is an integer of the form $2^n$,Where $n$ is a non-negative integer).
		\subsection*{geometry}
\item Problem 3. Let $ABC$ be an acute triangle with $AB\textgreater AC$ Let I be its circumcircle, $H$ its orthocentre, and $F$ the foot of the altitude from $A$. Let $M$ be the midpoint of $BC$. Let $Q$ he the point on $T$ such that $\angle HQA= 90$, and let $K$ be the point on $T$ such that $\angle HKQ=90\degree.$ Assume that the points$ A, B, C, K and Q$ are all different, and lie on $T$ in this order.\\

Prove that the circumcircles of triangles $KQH$ and $FKM$ are tangent to each other.
                \subsection*{geometry}
\item Problem 4. Triangle $ABC$ has circumcircle $\ohm$ and circumcentre $O$. A circle $T$ with centre. A intersects the segment $BC$ at points $D and E$, such that $B, D, E $and Care all different and lie on line $BC$ in this onter. Let $F and G $be the points of intersection of $T and \ohm$. such that $A. F B. C and G $lie on \ohm in this order. Let $K $ he the second point of intersection of the circumcircle of triangle $BDF$ and the segment $AB$. Let $L$ be the second point of intersection of the circumcircle of triangle $CGE$ and the segment $CA$\\
Suppose that the lines $FKand GL$ are different and intersect at the point $X$. Prove that $X$ lies on the line $AO$.
               \subsection*{geometry}

\item Problem 5. Let $R$ be the set of real numbers. Determine all functions $f:R\rightarrow R$ satisfying the equation
	\begin{align}
		f\brak{x+f\brak{x+y}}+f\brak{xy}=x+f\brak{x+y}+yf\brak{x}\\
	\end{align}
for all real numbers $x$ and $y$
\subsection*{geometry}
\item problem6 the sequence $a_1,a_2, \ldots$ of an integers satisfies the following conditions;
	\begin{enumerate}
		\item $1\leq a_{j} \leq2015$ for all $j\geq 1;$
		\item $k+a_{k} \neq l+a_{l}$ for all $1\leq k \textless l.$
	\end{enumerate}	
prove that there exist two positive integers $b and N$ such that\\

$\mydet {\sum_{j=m+1}^{n} \brak {aj-b} }\leq 1007^2$

for all integers $m and n$ satisfying $n\textgreater m\geq N$


\end{enumerate}
\end{document}
